\section{Automatsko testiranje}
Jedna od bitnih aktivnost u ciklusu je automatsko testiranje softvera. Automatsko testiranje značajno unapređuje razvojni proces. Ovakvi testovi daju sigurnost da nove izmene u kodu neće pogoršati stabilnost sistema, kao i da će funkcionalnosti koje su do tada radile, nastaviti da rade. Timovi treba da ulažu u pisanje testova jer se njima podiže kvalitet proizvoda, a cena testiranja je mnogo manja u odnosu na kasnije ispravke.

Najefikasniji pristup pisanja dobrih testova je da se testovi pišu na početku, pre pisanja samog k\^oda. Ovakva tehnika se naziva razvoj vođen testovima (eng.~{\em Test Driven Development}). U te svrhe se najčešće koriste testovi jedinica (eng.~{\em Unit tests}). 

U ovom poglavlju ćemo pokazati kako se u Jenkins može integrisati automatsko testiranje. Iako postoji mnogo različitih načina za testiranje softvera, držaćemo se uglavnom testova jedinica jer smatramo da je takav pristup testiranju najzastupljeniji.

\subsection{Uključivanje testova jedinica u proces}

